{\color{indiagreen}\subsection{Električna vezja}}
Električni tok teče po prevodnikih(kovini). Pomeni usmerjeno gibanje nabojev(pozitivnih in negativnih)\\
\begin{align*}
	I &= \frac{\Delta e}{\Delta t}[1A] \dots \text{kolikšen naboj preteče v določenem času}\\
\end{align*}
Prvič so električni tok opazovali v eelektrolizi modre galice.\\
%\begin{center}
%	\includegraphics[width=15cm, height=15cm,keepaspectratio=true]{ElTok.png}
%\end{center}
%\begin{center}
%	\includegraphics[width=15cm, height=15cm,keepaspectratio=true]{ElTok2.png}
%\end{center}
\textbf{Elektroni se gibljejo v nasprotno smer od električnega toka}\\
Tok teče samo po sklenjenem električnem krogu.\\
%\begin{center}
%	\includegraphics[width=15cm, height=15cm,keepaspectratio=true]{ElTok3.png}
%\end{center}
Učinki električnega toka:
\begin{itemize}
	\item Snov se segreje
	\item Prenaša se snov(elektroliza)
	\item magnetni(v okolici vodnika se pojavi magnetno polje)
\end{itemize}