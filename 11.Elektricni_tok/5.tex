{\color{indiagreen}\subsection{Vezave amper- in voltmetra}}
Merjenje toka z ampermetrom.\\
%\begin{center}
%	\includegraphics[width=15cm, height=15cm,keepaspectratio=true]{VezaveMetrov.png}
%\end{center}
Merjenje napetosti z voltmetrom.\\
%\begin{center}
%	\includegraphics[width=15cm, height=15cm,keepaspectratio=true]{VezaveMetrov2.png}
%\end{center}
\textbf{Merilno območje ampermetra}\\
%\begin{center}
%	\includegraphics[width=15cm, height=15cm,keepaspectratio=true]{VezaveMetrov3.png}
%\end{center}
\begin{align*}
	I_0 \dots \text{Največji tok(merilno območje)}\\
	I > I_0 \dots \text{Želimo meriti večje tokove od merilnega območja}\\
\end{align*}
%\begin{center}
%	\includegraphics[width=15cm, height=15cm,keepaspectratio=true]{VezaveMetrov4.png}
%\end{center}
\begin{align*}
	R_A &\dots \text{Notranji upor ampermetra}\\
	R &\dots \text{Soupor}\\
	U_a &= U_r\\
	R_a I_0 &= R(I - I_0)\\
	{\color{bostonuniversityred}R} = {\color{bostonuniversityred}\frac{R_a I_0}{I - I_0}}\\
\end{align*}
Če hočemo meriti tokove I, mortamo na ampermeter vezati upor ki je tako velik(kot kaže zgornja enačba)\\
\textbf{Merilno območje voltmetra}\\
%\begin{center}
%	\includegraphics[width=15cm, height=15cm,keepaspectratio=true]{VezaveMetrov4.png}
%\end{center}
\begin{align*}
	U_0 &\dots \text{Merilno območje}\\
	R &\dots \text{Predupor}\\
	U &> U_0\\
	R_v &\dots \text{notranji upor voltmetra}\\
	I_v &= I_r\\
	\frac{U_0}{R_v} &= \frac{U - U_0}{R}\\
	U_0 R &= R_v(U - U_0)\\
	{\color{bostonuniversityred}R} &= {\color{bostonuniversityred}\frac{R_v(U - U_0)}{U_0}}\\
\end{align*}
*padec napetost $\rightarrow$ napetosti na uporabniku\\