{\color{indiagreen}\subsection{Vezave uporov}}
\textbf{Zaporedna}\\
%\begin{center}
%	\includegraphics[width=15cm, height=15cm,keepaspectratio=true]{VezaveUporov.png}
%\end{center}
\begin{align*}
	I &= I_1 = I_2\\
	{\color{bostonuniversityred}U} &= {\color{bostonuniversityred}U_1 + U_2} \dots \text{2. kirchoffov zakon}\\
	RI &= RI_1 + RI_2\\
	{\color{bostonuniversityred}R} &= {\color{bostonuniversityred}R_1 + R_2}\\
	I_1 &= I_2\\
	\frac{U_1}{R_1} &= \frac{U_2}{R_2}\\
	{\color{bostonuniversityred}\frac{U_1}{U_2}} &= {\color{bostonuniversityred}\frac{R_1}{R_2}} \dots \text{Razmerje uporov}\\
\end{align*}
\textbf{Vzporedna}\\
%\begin{center}
%	\includegraphics[width=15cm, height=15cm,keepaspectratio=true]{VezaveUporov.png}
%\end{center}
\begin{align*}
	{\color{bostonuniversityred}I} &= {\color{bostonuniversityred}I_1 + I_2} \dots \text{1. kirchoffov zakon}\\
	U &= U_1 = U_2\\
	\frac{U}{R} &= \frac{U}{R_1} + \frac{U}{R_2}\\
	{\color{bostonuniversityred}\frac{1}{R}} &= {\color{bostonuniversityred}\frac{1}{R_1} + \frac{1}{R_2}}\\
	U_1 &= U_2\\
	R_1 I_1 &= R_2 I_2\\
	{\color{bostonuniversityred}\frac{I_1}{I_2}} &= {\color{bostonuniversityred}\frac{R_2}{R_1}} \dots \text{Razmerje tokov proti uporom. Večji je tok, manjši je upor.}\\
\end{align*}
\textbf{1. kirchoffov zakon}:Vsota tokov, ki priteče v razvejišče je enako vsoti tokov, ki odteče iz razvejišča\\