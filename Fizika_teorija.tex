\documentclass[a4paper,oneside,12pt]{article}
\usepackage[slovene]{babel}
\usepackage[utf8]{inputenc}
\usepackage{arev}
\usepackage[T1]{fontenc}
%\usepackage{color}
\usepackage{url}
\usepackage{graphicx}
\graphicspath{ {./Slike/} }
\usepackage[usenames]{color}
\usepackage[reqno]{amsmath}
\usepackage{amssymb}
\usepackage{enumerate}
\usepackage{array}
\usepackage[bookmarks, colorlinks=true, %
linkcolor=black, anchorcolor=black, citecolor=black, filecolor=black,%
menucolor=black, runcolor=black, urlcolor=black%
]{hyperref}
\usepackage[
    paper=a4paper,
    top=2.5cm,
    bottom=2.5cm,
%    textheight=24cm,
    textwidth=15cm,
    ]{geometry}

\usepackage{import}
\usepackage{icomma}
\usepackage{pgffor}
\usepackage{units}
\usepackage{minted}


\renewcommand{\familydefault}{\sfdefault} %drugace font

\newcommand*{\plogo}{\fbox{$\mathcal{PL}$}} % Generic publisher logo


%funkcija za tab
\usepackage{ifthen,xcolor}
\newlength{\tabcont}

\newcommand{\tab}[1]{%
\settowidth{\tabcont}{#1}%
\ifthenelse{\lengthtest{\tabcont < .25\linewidth}}%
{\makebox[.25\linewidth][l]{#1}\ignorespaces}%
{\makebox[.5\linewidth][l]{\color{red} #1}\ignorespaces}%
}%

\usepackage[makeroom]{cancel} % za crtanje texta
\usepackage{tikz}
\usepackage{pgfplots} %grafi
\pgfplotsset{
    standard/.style={
        every axis x label/.style={at={(current axis.right of origin)},anchor=north west}, %za dajanje labelov na x osi (yticklabels = {0, $v_0$})
        every axis y label/.style={at={(current axis.above origin)},anchor=north east} %za dajanje labelov na y osi
    }
}
\usepgfplotslibrary{fillbetween} %risanje po grafu
\usepackage{empheq} % za okvirjanje enačb
\usepackage{gensymb} %za stopinj celzije

\title{Fizika snov}
\author{Rok Kos}
\date{ \today}

%----------------------------------------------------------------------------------------
%	Naslovnica

\newcommand*{\titleGM}{\begingroup % Create the command for including the title page in the document
\hbox{ % Horizontal box
\hspace*{0.2\textwidth} % Whitespace to the left of the title page
\rule{1pt}{\textheight} % Vertical line
\hspace*{0.05\textwidth} % Whitespace between the vertical line and title page text
\parbox[b]{0.75\textwidth}{ % Paragraph box which restricts text to less than the width of the page

{\noindent\Huge\bfseries Fizika snov}\\[2\baselineskip] % Title
{\large \textit{Rok Kos}}\\[4\baselineskip] % Tagline or further description
%{\Large \textsc{\today}} % Author name

\vspace{0.5\textheight} % Whitespace between the title block and the publisher
{\noindent Gimnazija Vič, Tržaška cesta 72}\\[\baselineskip] % Publisher and logo
}}
\endgroup}

%----------------------------------------------------------------------------------------
%	Ostevilcevanje
\usepackage{fancyhdr}
\usepackage{lastpage}
 
\pagestyle{fancy}
\fancyhf{}
 
\rfoot{Page \thepage \hspace{1pt} of \pageref{LastPage}}
\lfoot{Rok Kos 4.c \textregistered \copyright }
\lhead{Gimnazija Vič}
\rhead{2015-2016}

%spreminjanje naslovov in podnaslov
\usepackage{titlesec}
\titleformat*{\section}{\LARGE\bfseries}
\titleformat*{\subsection}{\Large\bfseries}
\titleformat*{\subsubsection}{\large\bfseries}

%barve iz strani http://latexcolor.com/
\definecolor{indiagreen}{rgb}{0.07, 0.53, 0.03}
\definecolor{internationalorange}{rgb}{1.0, 0.31, 0.0}
\definecolor{alizarin}{rgb}{0.82, 0.1, 0.26}
\definecolor{bostonuniversityred}{rgb}{0.8, 0.0, 0.0}

%komanda za spreminjanje stevilk v rimske stevilke
\newcommand*{\RomanNumber}[1]{\expandafter\@slowromancap\romannumeral #1@}

\begin{document}
	%\pagestyle{empty} % Removes page numbers
	\titleGM
	\tableofcontents
	\newpage


	{\color{internationalorange}\section{FIZIKALNE KOLIČINE IN ENOTE}}
	\textbf{Fizikalna količina} je produkt merskega števila in merske enote. \\
	\\
	s = 5 m $\rightarrow$ \quad {\color{bostonuniversityred}{merska enota}}\\
	\hphantom{s =} $\downarrow$\\
	{\color{bostonuniversityred}{mersko št.}} 

	\foreach \n in {1,...,5}{
		\subimport{1.Fizikalne_kolicine_in_enote/}{\n.tex}
	}

	{\color{internationalorange}\section{PREMO IN KRIVO GIBANJE}}
	\foreach \n in {1,...,10}{
		\subimport{2.Premo_in_krivo_gibanje/}{\n.tex}
	}

	{\color{internationalorange}\section{SILA IN NAVOR}}
	\foreach \n in {1,...,10}{
		\subimport{3.Sila_in_navor/}{\n.tex}
	}

	{\color{internationalorange}\section{NEWTNOVI ZAKONI IN GRAVITACIJA}}
	\foreach \n in {1,...,2}{
		\subimport{4.Newtnovi_zakoni_in_gravitacija/}{\n.tex}
	}
	{\color{internationalorange}\section{IZREK O GIBALNI KOLIČINI}}
	%\foreach \n in {1,...,2}{
		\subimport{5.Izrek_o_gibalni_kolicini/}{1.tex}
	%}

	{\color{internationalorange}\section{DELO IN ENERGIJA}}
	\foreach \n in {1,...,8}{
		\subimport{6.Delo_in_energija/}{\n.tex}
	}

	{\color{internationalorange}\section{TEKOČINA}}
	\foreach \n in {1,...,2}{
		\subimport{7.Tekocine/}{\n.tex}
	}

	{\color{internationalorange}\section{TEMPERATURA}}
	\foreach \n in {1,...,5}{
		\subimport{8.Temperatura/}{\n.tex}
	}

	{\color{internationalorange}\section{NOTRANJA ENERGIJA IN TOPLOTA}}
	\foreach \n in {1,...,7}{
		\subimport{9.Notranja_energija_in_toplota/}{\n.tex}
	}

	{\color{internationalorange}\section{ELEKTRIČNI NABOJ IN ELEKTRIČNO POLJE}}
	\foreach \n in {1,...,9}{
		\subimport{10.El_naboj_in_el_polje/}{\n.tex}
	}

	{\color{internationalorange}\section{ELEKTRIČNI TOK}}
	\foreach \n in {1,...,6}{
		\subimport{11.Elektricni_tok/}{\n.tex}
	}

	{\color{internationalorange}\section{MAGNETNO POLJE}}
	\foreach \n in {1,...,5}{
		\subimport{12.Magnetno_polje/}{\n.tex}
	}

	{\color{internationalorange}\section{INDUKCIJA}}
	\foreach \n in {1,...,9}{
		\subimport{13.Indukcija/}{\n.tex}
	}
\end{document}