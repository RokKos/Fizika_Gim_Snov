{\color{indiagreen}\subsection{Trajni magneti}}
Magnetno polje nastane v 3 primerih:
\begin{itemize}
	\item V okolici trajnega magneta
	\item Vsi vodniki po katerih teče tok
	\item Mešanica obojega(elektro magnet)
\end{itemize}
\textbf{Trajni magnet}\\
\begin{itemize}
	\item je iz, ki ima \textbf{feromagnetno strukturo}
	\item v naravi je 5 elementov: železo, nikelj, kobalt, gadolinij(gd), disprozij(dy)
	\item Vsak magnet ima južni in severni pol
	\item Nikoli ne moremo imeti ločenega enega pola(tudi če magnet razdelimo)
	\item Magnetne silnice potekajo tudi znotraj magneta, zato jim pravimo, da so zaključne krivulje
	\item Silnice kažejo od severnega pola k južnemu polu
\end{itemize}
%\begin{center}
%	\includegraphics[width=15cm, height=15cm,keepaspectratio=true]{MagnetnoPolje.png}
%\end{center}
%\begin{center}
%	\includegraphics[width=15cm, height=15cm,keepaspectratio=true]{MagnetnoPolje2.png}
%\end{center}
\textbf{Zemlja}\\
%\begin{center}
%	\includegraphics[width=15cm, height=15cm,keepaspectratio=true]{MagnetnoPolje2.png}
%\end{center}
Severni geografski pol(južni magnetni pol) in obratno\\
Magnetne sile (na daljavo):
\begin{itemize}
	\item Privlačne med različnimi poli
	\item Odbojne med istoimenskimi poli
\end{itemize}
\textbf{Vodnik s tokom}\\
%\begin{center}
%	\includegraphics[width=15cm, height=15cm,keepaspectratio=true]{MagnetnoPolje3.png}
%\end{center}
{\color{bostonuniversityred}Pravilo desnega palca}\\
\textbf{Tuljava}\\
%\begin{center}
%	\includegraphics[width=15cm, height=15cm,keepaspectratio=true]{MagnetnoPolje4.png}
%\end{center}
%\begin{center}
%	\includegraphics[width=15cm, height=15cm,keepaspectratio=true]{MagnetnoPolje5.png}
%\end{center}
{\color{bostonuniversityred}Homogenost polja}\\
\textbf{Elektro motor}\\
%\begin{center}
%	\includegraphics[width=15cm, height=15cm,keepaspectratio=true]{MagnetnoPolje6.png}
%\end{center}
Ko tuljavo namagnetimo(skozi gre tok), se namagneti tudi železno jedro in se magnetno polje sešteva. So najmočnejši magneti. Potrebujejo energijo, da gre skozi tuljavo tok.\\
