{\color{indiagreen}\subsection{Vzgon}}
%\begin{center}
%	\includegraphics[width=15cm, height=15cm,keepaspectratio=true]{Vzgon.png}
%\end{center}
Telo potopljeno v kaplevino\\
%\begin{center}
%	\includegraphics[width=15cm, height=15cm,keepaspectratio=true]{Vzgon2.png}
%\end{center}
Vzgon je rezultanta sil okoliške kaplevine na potopljeno telo in prijemališče ima v težišču izpodrinjene kapljevine. Sila vzgona je po velikosti enaka teži izpodrinjene kapljevine. \\
\begin{align*}
	F_{vzg} &= \rho Vg \text{gostota kapljevine in volumen izpodrinjene kapljevine}\\
\end{align*}
\textbf{Telo plava}
%\begin{center}
%	\includegraphics[width=15cm, height=15cm,keepaspectratio=true]{Vzgon3.png}
%\end{center}
$\rho_{telo} < \rho_{kaplevina}$
\textbf{Telo lebdi}
%\begin{center}
%	\includegraphics[width=15cm, height=15cm,keepaspectratio=true]{Vzgon4.png}
%\end{center}
$\rho_{telo} = \rho_{kaplevina}$
\textbf{Telo potone}
%\begin{center}
%	\includegraphics[width=15cm, height=15cm,keepaspectratio=true]{Vzgon4.png}
%\end{center}
$\rho_{telo} > \rho_{kaplevina}$