{\color{indiagreen}\subsection{Lastni nihanji časi nihal}}
\textbf{Vzmetno nihalo}(izpeljavo moraš znat!!!)\\
%\begin{center}
%	\includegraphics[width=15cm, height=15cm,keepaspectratio=true]{LastNihCas.png}
%\end{center}
*brez trenja\\
\begin{align*}
	F_v & = -k s_0 \dots \text{po hookovem zakonu}\\
	F &= ma_0 \dots \text{po njutnovem zakonu}\\
	a _0 &= -\omega^2 s_0\\
	-k \cancelto{}{s_0} &= -\omega^2 \cancelto{}{s_0} m\\
	{\color{bostonuniversityred}\omega} &= {\color{bostonuniversityred}\sqrt{\frac{k}{m}}}\\
	\omega &= \frac{2 \pi}{t_0}\\
	{\color{bostonuniversityred}t_0} &= {\color{bostonuniversityred}2 \pi \sqrt{\frac{m}{k}}}\\
\end{align*}
\begin{tikzpicture}
	\begin{axis}[
	    xlabel={m},
	    ylabel={$t_0$},
	    xmin=0, xmax=10,
	    ymin=0, ymax=10,
	    xtick={0,2,4,6,8,10},
	    ytick={0,2,4,6,8,10},
	    ymajorgrids=true,
	    xmajorgrids=true,
	    grid style=dashed,
	    axis lines=middle,
	]
	\addplot[domain=0:10,red] {sqrt(3*x)};
	\end{axis}
\end{tikzpicture}\\
\begin{tikzpicture}
	\begin{axis}[
	    xlabel={$\frac{1}{k}$},
	    ylabel={$t_0$},
	    xmin=0, xmax=10,
	    ymin=0, ymax=10,
	    xtick={0,2,4,6,8,10},
	    ytick={0,2,4,6,8,10},
	    ymajorgrids=true,
	    xmajorgrids=true,
	    grid style=dashed,
	    axis lines=middle,
	]
	\addplot[domain=0:10,red] {sqrt(3*x)};
	\end{axis}
\end{tikzpicture}\\
\begin{tikzpicture}
	\begin{axis}[
	    xlabel={$s_0$},
	    ylabel={$t_0$},
	    xmin=0, xmax=10,
	    ymin=0, ymax=10,
	    xtick={0,2,4,6,8,10},
	    ytick={0,2,4,6,8,10},
	    ymajorgrids=true,
	    xmajorgrids=true,
	    grid style=dashed,
	    axis lines=middle,
	]
	\addplot[domain=0:10,red] {4};
	\end{axis}
\end{tikzpicture}\\
\textbf{Matematično nihalo}\\
%\begin{center}
%	\includegraphics[width=15cm, height=15cm,keepaspectratio=true]{LastNihCas.png}
%\end{center}
*Nihanje matematičnega nihala ni \textbf{sinusno}, ker sila \textbf{ni sorazmerna z odmikom}. V  tem primeru je sila sorazmerna s sinusom\\
\begin{align*}
	F &= F_g \sin \rho_0\\
	\sin \rho_0 = \rho_0 \dots &\text{\textbf{Za majne kote $\rho_0$}, vse od zdaj naprej delamo na tej podlagi.}\\
	F &= mg \rho_0\\
	\cancelto{}{m}g \rho_0 &= \cancelto{}{m} \omega^2 s_0\\
	l &= r \rho\\
	s_0 &= l\rho\\
	g \rho_0 &= \omega^2 l \rho_0\\
	{\color{bostonuniversityred}\omega} &= {\color{bostonuniversityred}\sqrt{\frac{g}{l}}}\\
	{\color{bostonuniversityred}t_0} &= {\color{bostonuniversityred}2 \pi \sqrt{\frac{l}{g}}}\\
\end{align*}
\begin{tikzpicture}
	\begin{axis}[
	    xlabel={l},
	    ylabel={$t_0$},
	    xmin=0, xmax=10,
	    ymin=0, ymax=10,
	    xtick={0,2,4,6,8,10},
	    ytick={0,2,4,6,8,10},
	    ymajorgrids=true,
	    xmajorgrids=true,
	    grid style=dashed,
	    axis lines=middle,
	]
	\addplot[domain=0:10,red] {sqrt(3*x)};
	\end{axis}
\end{tikzpicture}\\
\begin{tikzpicture}
	\begin{axis}[
	    xlabel={m},
	    ylabel={$t_0$},
	    xmin=0, xmax=10,
	    ymin=0, ymax=10,
	    xtick={0,2,4,6,8,10},
	    ytick={0,2,4,6,8,10},
	    ymajorgrids=true,
	    xmajorgrids=true,
	    grid style=dashed,
	    axis lines=middle,
	]
	\addplot[domain=0:10,red] {3};
	\end{axis}
\end{tikzpicture}\\
\begin{tikzpicture}
	\begin{axis}[
	    xlabel={$s_0$},
	    ylabel={$t_0$},
	    xmin=0, xmax=10,
	    ymin=0, ymax=10,
	    xtick={0,2,4,6,8,10},
	    ytick={0,2,4,6,8,10},
	    ymajorgrids=true,
	    xmajorgrids=true,
	    grid style=dashed,
	    axis lines=middle,
	]
	\addplot[domain=0:10,red] {6};
	\end{axis}
\end{tikzpicture}\\
\textbf{*Nihanji čas je neodvisen od amplitude, ker so majhni koti.}\\