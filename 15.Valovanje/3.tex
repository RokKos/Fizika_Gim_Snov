{\color{indiagreen}\subsection{Stoječe valovanje}}
\textbf{Oboj motnje}:
\begin{itemize}
	\item Vpet konec vrvi\\
		%\begin{center}
		%	\includegraphics[width=15cm, height=15cm,keepaspectratio=true]{OdbojValovanja.png}	
		%\end{center}
		obojo z nasprotno fazo
	\item Prost konec vrvi\\
		%\begin{center}
		%	\includegraphics[width=15cm, height=15cm,keepaspectratio=true]{OdbojValovanja2.png}	
		%\end{center}
		odboj z isto fazo
\end{itemize}
\textbf{Vsota vpadnega in odbojnega kota}:
%\begin{center}
%	\includegraphics[width=15cm, height=15cm,keepaspectratio=true]{OdbojValovanja3.png}	
%\end{center}
%\begin{center}
%	\includegraphics[width=15cm, height=15cm,keepaspectratio=true]{OdbojValovanja4.png}	
%\end{center}
\begin{align*}
	V &\dots \text{Vozel stoječega valovanja}\\
	h &\dots \text{Hrbet stoječega valovanja}\\
\end{align*}
Vsi deli vrvi nihajo z enako frekvenco, ampak z različno amplitudo.