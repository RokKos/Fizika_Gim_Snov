{\color{indiagreen}\subsection{Lastno nihanje strune}}
%\begin{center}
%	\includegraphics[width=15cm, height=15cm,keepaspectratio=true]{LastnoNihanjeStrune.png}	
%\end{center}
\begin{align*}
	\nu &= \frac{c}{\lambda}\\
	e &= \frac{\lambda_0}{2}\\
	\nu_0 &= \frac{c}{\lambda_0} = \frac{c}{2l}\\
	{\color{bostonuniversityred}\nu_0} = {\color{bostonuniversityred}\frac{c}{2l}} &\dots \text{Osnovna lastna frekvenca, ki jo lahko struna odda.}\\
	\nu_1 = \frac{c}{\lambda_1}\\
	\lambda_1 &= l\\
	{\color{bostonuniversityred}\nu_1 = \frac{c}{l}} &= {\color{bostonuniversityred}2\nu_0} \dots \text{Prva višjeharmonična lastna frekvenca.}\\
	\nu_2 = \frac{c}{\lambda_2}\\
	l &= \frac{3\lambda_2}{2}\\
	\lambda_2 &= \frac{2l}{3}\\
	\nu_2 &= \frac{3c}{2l} = 3\nu_0\\
	{\color{bostonuniversityred}\nu_n} &= {\color{bostonuniversityred}n\nu_0}\\
\end{align*}