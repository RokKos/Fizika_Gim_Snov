{\color{indiagreen}\subsection{Valovanje na vodni površini}}
\textbf{Krožno valovanje}\\
%\begin{center}
%	\includegraphics[width=15cm, height=15cm,keepaspectratio=true]{ValovanjeVodnaPovrsina.png}	
%\end{center}
$\lambda$ \dots valovna dolžina.\\
Valovna črta povezuje sosednje točke v ravnini, ki so v danem trenutku enako oddaljene od ravnovesne lege.\\
Valovni žarki so vedno pravokotni na valovne črte.\\
\textbf{Ravno valovanje}\\
%\begin{center}
%	\includegraphics[width=15cm, height=15cm,keepaspectratio=true]{ValovanjeVodnaPovrsina2.png}	
%\end{center}
Črte so med seboj ravne in vzporedne.\\
\textbf{VALOVNI POJAVI}\\
\begin{enumerate}
	\item ODBOJ\\
		%\begin{center}
		%	\includegraphics[width=15cm, height=15cm,keepaspectratio=true]{ValovanjeVodnaPovrsina3.png}	
		%\end{center}
		\begin{align*}
			\alpha &\dots \text{vpadni kot}\\
			\beta &\dots \text{odbojni kot}\\
			{\color{bostonuniversityred}\alpha} &= {\color{bostonuniversityred}\beta} \dots \text{Odbojni zakon}\\
		\end{align*}
		Ohranijo se vse količine(frekvenca, hitrost, valovna dolžina) razen \textbf{smer hitrosti}.\\
	\item LOM\\
		Kadar prehaja valovanje iz enega sredstva v drugo.\\
		%\begin{center}
		%	\includegraphics[width=15cm, height=15cm,keepaspectratio=true]{ValovanjeVodnaPovrsina3.png}	
		%\end{center}
		\begin{align*}
			\alpha &\dots \text{vpadni kot}\\
			\beta &\dots \text{odbojni kot}\\
			\sin\alpha &= \frac{\lambda_1}{l}\\
			\sin\beta &= \frac{\lambda_2}{l}\\
			\frac{\sin\alpha}{\sin\beta} &= \frac{\frac{\lambda_1}{l}}{\frac{\lambda_2}{l}} = \frac{\lambda_1}{\lambda_2}\\
			c &= \lambda \nu\\
			\lambda &= \frac{c}{\nu} \dots \text{Frekvenca ostane enaka.}\\
			\frac{\sin\alpha}{\sin\beta} &= \frac{c_1 \cancelto{}{\nu}}{c_2 \cancelto{}{\nu}}\\
			{\color{bostonuniversityred}\frac{\sin\alpha}{\sin\beta}} &= {\color{bostonuniversityred}\frac{c_1}{c_2}} \dots \text{Lomni zakon}\\
			c_1 &\dots \text{Vpadna hitrost}\\
			c_2 &\dots \text{Lomna hitrost}\\
		\end{align*}
		Pri lomu se spremeni vse \textbf{razen frekvence}, spremenijo se valovna dolžina, smer in velikost hitrosti.\\
		\begin{align*}
			\frac{1}{c_1}\sin\alpha &= \frac{1}{c_2}\sin\beta /*c_0 \dots \text{Hitrost svetlobe v vakumu}\\
			c_0 &= \unitfrac[3*10^8]{\metre}{\second}\\
			\frac{c_0}{c_1}\sin\alpha &= \frac{c_0}{c_2}\sin\beta\\
			{\color{bostonuniversityred}n_1 \sin\alpha} &= {\color{bostonuniversityred}n_2 \sin\beta}\\
			n_1 &\dots \text{Lomni količnik}\\
			n_2 &\dots \text{Lomni količnik}\\
			{\color{bostonuniversityred}\frac{\sin\alpha}{\sin\beta}} &= {\color{bostonuniversityred}\frac{n_2}{n_1}} \dots \text{Lomni zakon}\\
			n &= \frac{\text{hitrost svetlobe v vakumu}}{\text{hitrost svetlobe v snovi}} = \frac{c_0}{c_s}\\
			{\color{bostonuniversityred}n} &= {\color{bostonuniversityred}\frac{c_0}{c_s}} \geq 1\\
			n_{zrak} &= 1\\
			n_{vode} &= 1,33\\
			n_{steklo} &= 1,5\\
		\end{align*}
		Večji kot je lomni količnik bolj se žarek lomi.\\
		Ista snov \textbf{lomi različne valovne dolžine različno}.\\
	\item POPOLNI ODBOJ\\
		%\begin{center}
		%	\includegraphics[width=15cm, height=15cm,keepaspectratio=true]{ValovanjeVodnaPovrsina4.png}	
		%\end{center}
		\begin{align*}
			\beta > \alpha\\
			n_{voda} > n_{zrak}\\
		\end{align*}
		Vedno, ko gre za prehod iz snovi z večjim lomnim količnikom v manjšega, je lomni kot večji od vpadnega.\\
		%\begin{center}
		%	\includegraphics[width=15cm, height=15cm,keepaspectratio=true]{ValovanjeVodnaPovrsina5.png}	
		%\end{center}
		\begin{align*}
			\beta < \alpha\\
			n_{voda} < n_{zrak}\\
		\end{align*}
		%\begin{center}
		%	\includegraphics[width=15cm, height=15cm,keepaspectratio=true]{ValovanjeVodnaPovrsina6.png}	
		%\end{center}
\end{enumerate}