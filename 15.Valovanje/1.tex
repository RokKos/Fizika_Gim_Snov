{\color{indiagreen}\subsection{Vrste valovanja}}
Valovanje je sestavljeno nihanje.\\
Poznamo 2 vrsti valovanja:
\begin{itemize}
	\item Longitudinalno(vzdolžno) valovanje npr. zvok
	%\begin{center}
	%	\includegraphics[width=15cm, height=15cm,keepaspectratio=true]{VrsteValovanja.png}
	%\end{center}
	Značilnost je, da so odmiki sredstva vzporedni z smerjo širjenja valovanja.
	\item Transferzalno(prečno) valovanje npr. elektromagnetno valovanje
	%\begin{center}
	%	\includegraphics[width=15cm, height=15cm,keepaspectratio=true]{VrsteValovanja2.png}
	%\end{center}
	Motnje so htibi in doline.\\
	Značilnost je da so odmiki sredstva pravokotni na smer širjenja valovanja.
\end{itemize}

\textbf{Trenutna slika valovanja}
%\begin{center}
%	\includegraphics[width=15cm, height=15cm,keepaspectratio=true]{VrsteValovanja2.png}	
%\end{center}
\begin{align*}
	\lambda &\dots \text{valovna dolžina(razdalja med dvema sosednjima deloma, ki nihata sočasno)}\\
	\text{Vir valovanja: } & \nu, t_0\\
	s &= vt\\
	{\color{bostonuniversityred}\lambda} &= {\color{bostonuniversityred}c * t_0}\\
	c &\dots \text{hitrost motnje}\\
	t_0 &= \frac{1}{\nu}\\
	{\color{bostonuniversityred}c} &= {\color{bostonuniversityred}\lambda \nu}\\
\end{align*}
Če spremenimo frekvenco \textbf{se ne spremeni hitrost}, ampak je konstantna, \textbf{spremeni se valovna dolžina}.
