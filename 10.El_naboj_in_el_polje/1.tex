{\color{indiagreen}\subsection{Električni naboj}}
\textbf{Atom:}
\begin{itemize}
    \item jedro
    \item električni ovoj(negativen naboj)
\end{itemize}
\textbf{Naboj:}
\begin{itemize}
    \item negativni($e^-$)
    \item pozitivni($p^+$)
\end{itemize}
Električno nevralno telo je, če ima enako negativnega in pozitivnega naboja. Naelektreno telo ima presežke ene vrste naboja.\\
\begin{align*}
    e &\dots \text{naboj(kvantiziran, del nečesa, ki ga se neznamo dati na manjše dele)}\\
    e_0 &= 1.6 * 10^{-19}\dots \text{osnovni naboj(naboj elektrona $e^-$)}\\
    e &= ne_0; n € Z popravi \\
\end{align*}
Sila med naboji(sila na daljavo):
\begin{itemize}
    \item odbojna(med istoimenskimi naboji)
    \item privlačna(med razboimeskimi naboji)
\end{itemize}
Snovi:
\begin{itemize}
    \item prevodniki(kovine)
    \item izolatorji
\end{itemize}
Naboj merimo z \textbf{elektroskopom}. Ne da se ugotoviti kako je telo nabito, lahko samo ugotovimo, da je ali ni.
%\begin{center}
%   \includegraphics[width=15cm, height=15cm,keepaspectratio=true]{JakostElNaboja.png}
%\end{center}