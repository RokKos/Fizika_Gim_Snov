{\color{indiagreen}\subsection{Kondenzator}}
Kondenzator je naprava shranjevanje naboja. Obravnavali bomo ploščati kondenzator pri katerem je ena plošča pozitivna druga pa negativno nabita.\\
\begin{align*}
	C &= \frac{e}{U}[1\frac{As}{V} = 1\frac{c}{V} = 1F] \dots \text{fahrad}\\
	C &\dots \text{kapaciteta kondenzatorja(koliko naboja lahko shranimo)}\\
	C &= \frac{\varepsilon_0 S}{d}\\
	U &= Ed\\
	U &= \frac{e}{C} = \frac{ed}{\varepsilon_0 S}\\
	{\color{bostonuniversityred}E} &= {\color{bostonuniversityred}\frac{e}{\varepsilon_0 S}} \dots \text{električno polje med ploščama}\\
	{\color{bostonuniversityred}E} &= {\color{bostonuniversityred}\frac{e}{2\varepsilon_0 S}} \dots \text{električno polje v okolici ene nabite plošče}\\
\end{align*}
%\begin{center}
%	\includegraphics[width=15cm, height=15cm,keepaspectratio=true]{Kondezator.png}
%\end{center}