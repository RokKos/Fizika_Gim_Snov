{\color{indiagreen}\subsection{Električna napetost}}
%\begin{center}
%	\includegraphics[width=15cm, height=15cm,keepaspectratio=true]{ElektricnaNapetost.png}
%\end{center}
\begin{align*}
	A &= F^{'}_e s\\
	F^{'}_e &= F_e cos\alpha = e E cos\alpha\\
	A &= e E cos\alpha s\\
	cos\alpha &= \frac{h}{s}\\
	A &= e E \frac{h}{s} s\\
	{\color{bostonuniversityred}A} &= {\color{bostonuniversityred}e E h}\\
	{\color{bostonuniversityred}U} &= {\color{bostonuniversityred}E h}\\
	U = \frac{A}{e}[1\frac{J}{c} &= 1\frac{J}{As} = 1V] \dots \text{električna napetost}\\
	1J &= 1 VAs\\ 
	{\color{bostonuniversityred}A} &= {\color{bostonuniversityred}e U}\\
\end{align*}
Električna napetost nam pove kolikšno delo opravimo na enoto naboja v tem električnem polju.\\
Prenos: $1 \rightarrow 3 \rightarrow 2$
\begin{align*}
	A &= A_1 + A_2\\
	eU &= eU_1 + eU_2\\
	{\color{bostonuniversityred}U} &= {\color{bostonuniversityred}U_1 + U_2}\\ 
\end{align*}
\textbf{Električni potencial V[V]}
$U = V_2 - V_1$ potencial v smeri silnic pada
%\begin{center}
%	\includegraphics[width=15cm, height=15cm,keepaspectratio=true]{ElektricnaNapetost2.png}
%\end{center}
{\color{bostonuniversityred} Ekvipotencialna ploskev} sestavljajo sosednje točke v prostoru, ki imajo enak potencial.\\
\textbf{Točkasti naboj}
%\begin{center}
%	\includegraphics[width=15cm, height=15cm,keepaspectratio=true]{ElektricnaNapetost3.png}
%\end{center}
\textbf{Homogeno polje}
%\begin{center}
%	\includegraphics[width=15cm, height=15cm,keepaspectratio=true]{ElektricnaNapetost4.png}
%\end{center}
