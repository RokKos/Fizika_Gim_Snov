{\color{indiagreen}\subsection{Jakost električnega polja}}
%\begin{center}
%	\includegraphics[width=15cm, height=15cm,keepaspectratio=true]{JakostElNaboja.png}
%\end{center}
\begin{align*}
    F &= e_1 \overline{E}\\
    E &\dots \text{jakost električnega polja}\\
    \overline{E} &= \frac{\overline{F}}{e_1}\\
\end{align*}
Dogovor: smer jakosti električne je enaka smeri sile na pozitivni naboj.\\
%\begin{center}
%	\includegraphics[width=15cm, height=15cm,keepaspectratio=true]{JakostElNaboja2.png}
%\end{center}
Gostota silnic je merilo za jakost električnega naboja.\\
\textbf{Točkasti naboj}
\begin{align*}
    F &= \frac{e e_1}{4 \pi \sigma_0 r^2}\\
    F &= e_1 E\\
    {\color{bostonuniversityred}E} &= {\color{bostonuniversityred}\frac{e}{4 \pi \sigma_0 r^2}}\\
\end{align*}
%\begin{center}
%	\includegraphics[width=15cm, height=15cm,keepaspectratio=true]{JakostElNaboja3.png}
%\end{center}
%\begin{center}
%	\includegraphics[width=15cm, height=15cm,keepaspectratio=true]{JakostElNaboja4.png}
%\end{center}
%\begin{center}
%	\includegraphics[width=15cm, height=15cm,keepaspectratio=true]{JakostElNaboja5.png}
%\end{center}
\textbf{Nasprotno enaki nabiti plošči}
%\begin{center}
%	\includegraphics[width=15cm, height=15cm,keepaspectratio=true]{JakostElNaboja5.png}
%\end{center}
\textbf{Homogeno električno polje}(z ravnimi medseboj vzporednimi silnicami)
%\begin{center}
%	\includegraphics[width=15cm, height=15cm,keepaspectratio=true]{JakostElNaboja5.png}
%\end{center}