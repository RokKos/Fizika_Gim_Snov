{\color{indiagreen}\subsection{Colombov zakon}}
%\begin{center}
%	\includegraphics[width=15cm, height=15cm,keepaspectratio=true]{ColombovZakon.png}
%\end{center}
\begin{align*}
    {\color{bostonuniversityred}F} &= {\color{bostonuniversityred}\frac{e_1 e_2}{4\pi \sigma_0 r^2}}\\
    \sigma_0 &= 8.85 * 10^{-12} \frac{As}{Vm}[\frac{C^2}{Nm^2}]\\
\end{align*}
Če naboj povečamo, se obe sili povečata.\\
\begin{tikzpicture}
	\begin{axis}[
	    xlabel={r},
	    ylabel={F},
	    xmin=0, xmax=10,
	    ymin=-0, ymax=10,
	    xtick={0,2,4,6,8,10},
	    ytick={0,2,4,6,8,10},
	    ymajorgrids=true,
	    xmajorgrids=true,
	    grid style=dashed,
	    axis lines=middle,
	]
	\addplot[domain=0:10,red] {5/(x)};
	\end{axis}
\end{tikzpicture}
%\begin{center}
%	\includegraphics[width=15cm, height=15cm,keepaspectratio=true]{ColombovZakon2.png}
%\end{center}
Če je več nabojev upoštevamo vse in jih vektorsko seštejemo.\\
Med središčema se naboj porazdeli po površini.
%\begin{center}
%	\includegraphics[width=15cm, height=15cm,keepaspectratio=true]{ColombovZakon3.png}
%\end{center}