{\color{indiagreen}\subsection{Snov v električnem polju}}
\textbf{Kovina}:
%\begin{center}
%	\includegraphics[width=15cm, height=15cm,keepaspectratio=true]{SnovVElektricnemPolju.png}
%\end{center}
silnice so pravokotne.\\
Zaradi prerazporeditve elektronov znotraj krogle, notri ni električnega polja.Temu pravimo \textbf{influeca}. Uporablja se pri ločevanju nabojev.\\
%\begin{center}
%	\includegraphics[width=15cm, height=15cm,keepaspectratio=true]{SnovVElektricnemPolju.png}
%\end{center}
Damo narazen in dobimo eno negativno in eno pozitivno ploščo.\\
\textbf{Izolator}:
%\begin{center}
%	\includegraphics[width=15cm, height=15cm,keepaspectratio=true]{SnovVElektricnemPolju.png}
%\end{center}
$E > 0$ znotraj je polje ampak je oslabljeno oz. manjše kot zunaj.\\
\textbf{Dialektrik}:
%\begin{center}
%	\includegraphics[width=15cm, height=15cm,keepaspectratio=true]{SnovVElektricnemPolju.png}
%\end{center}