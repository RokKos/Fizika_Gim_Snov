{\color{indiagreen}\subsection{Navpični met navzgor}}

\textbf{GRAFI:}\\
\begin{tikzpicture}
\begin{axis}[
    xlabel={t[s]},
    ylabel={a[$\frac{m}{s^2}$]},
    xmin=0, xmax=10,
    ymin=-10, ymax=10,
    xtick={0,2,4,6,8,10},
    ytick={-10,-5,0, 5, 10},
    yticklabels={-10,$-g$,0, $g$,10},
    ymajorgrids=true,
    xmajorgrids=true,
    grid style=dashed,
]
 
\addplot[domain=0:10,red] {-5};
 
\end{axis}
\end{tikzpicture}

{\color{bostonuniversityred}Smer in velikost pospeška sta vedno ista(osvisna od mase zemlje.)} Ko gre telo gor govorimo o pojemku, ko pa dol pa o pospešku.

\begin{tikzpicture}
\begin{axis}[
    xlabel={t[s]},
    ylabel={v[$\frac{m}{s}$]},
    xmin=0, xmax=10,
    ymin=-10, ymax=10,
    xtick={0,2,4,6,8,10},
    ytick={-10,-5,0, 5, 10},
    yticklabels={$-v_0$,-5,0, 5,$v_0$},
    ymajorgrids=true,
    xmajorgrids=true,
    grid style=dashed,
]
 
\addplot[domain=0:10,red] {-2*x+10};
 
\end{axis}
\end{tikzpicture}

Ker je pospešek vedno enak se graf ne lomi.

\begin{tikzpicture}
\begin{axis}[
    xlabel={t[s]},
    ylabel={v[$\frac{m}{s}$]},
    xmin=0, xmax=10,
    ymin=0, ymax=10,
    xtick={0,2,4,6,8,10},
    ytick={0,2,4,5,6,8,10},
   	yticklabels={0,2,4,$h$,6,8,$2h$},
    ymajorgrids=true,
    xmajorgrids=true,
    grid style=dashed,
]
 
\addplot[domain=0:10,red, restrict y to domain=-100:100] {(1/25) * ((x-5)^3)+5};
 
\end{axis}
\end{tikzpicture}\\
\textbf{ENAKOMERNO POJEMAJOČE}\\
\begin{align*}
	v &= v_0 \pm gt\\
	h &= v_0 t \pm \frac{gt^2}{2}\\
	v^2 &= v_0^2 \pm 2gh\\
\end{align*}
\textbf{ENAKOMERNO POSPEŠUJOČE}\\
\begin{align*}
	v &=gt\\
	h &= \frac{gt^2}{2}\\
	v^2 &= 2gh\\
\end{align*}