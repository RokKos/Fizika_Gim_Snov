{\color{indiagreen}\subsection{Splošna plinska enačba}}
%\begin{center}
%	\includegraphics[width=15cm, height=15cm,keepaspectratio=true]{Temperaturno_raztezanje_snovi.png}
%\end{center}
Okrogla posoda, molekule trkajo ob stene in ustvarjajo tlak\\
\begin{align*}
	n &\dots \text{molekul idealnega plina(število)}\\
	r &\dots \text{polmer posode}\\
	p_1 &= \frac{F}{s} \text{tlak, ki ga ustvari ena mulekula}\\
	p &= N\frac{F}{s}\\
	F &= \mu a_r \text{$\mu \dots$ masa ene mulekule}\\
	a_r &= \frac{\overline{v}^2}{r}\\
	S &= 4\pi r^2\\
	p &= N\frac{\mu\overline{v}^2}{4\pi r^3} * \frac{3}{3}\\
	p &= \frac{N\mu\overline{v}^2}{3V}\\
	\overline{W_k} &=\frac{\mu\overline{v}^2}{2} &= \frac{3}{2}kT\\
	\mu\overline{v}^2 &= 3kT\\
	p &= \frac{N\cancelto{}{3}kT}{\cancelto{}{3}V}\\
	{\color{bostonuniversityred}pV} &= {\color{bostonuniversityred}NkT}\text{Splošna plinska enačna}\\
	N &=N_a * n\\
	N_a = 6,02*10^{23} mol^{-1} &= 6,02*10^{20} kmol^{-1}\dots \text{avogadrovo število}\\ 
	pV &= n N_a kT\\
	N_a k = {\color{bostonuniversityred}R} &={\color{bostonuniversityred} 8310 \frac{J}{K kmol}}\\
	{\color{bostonuniversityred}pV} &= {\color{bostonuniversityred}nRT}\text{temperatura zmeraj v kelvinih}\\
\end{align*}