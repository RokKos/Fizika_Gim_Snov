{\color{indiagreen}\subsection{Temperatura}}
Temperatura je količina, ki opisuje stanje snovi.\\
Je neurejeno termično gibanje, molekule se vedno premikajo in višja je temperatura bolj se gibljejo, odvisno je tudi od kemične vezi.\\
S tem se je ukvarjal Ludwig Edward Boltzmann.\\
\begin{align*}
	{\color{bostonuniversityred}\overline{W_k}} &= {\color{bostonuniversityred}\frac{3}{2}kT} \text{temperatura obvezno v kelvinih}\\ 
	k &\dots \text{Boltzmannova konstanta}\\
	k &= 1,38 * 10 ^ {-23} \frac{J}{K}\\
	W_k &\dots \text{Povprečna kinetična energija molekule}\\
	T &\dots \text{temperatura} [^{\circ} C, K]\\
\end{align*}
Celzijeva skala $\rightarrow$ ledišče vode $0^{\circ} C$, vrelišče vode $100^{\circ} C$\\
Kelvinova skala na osnovi krčenja plinov. Ta lestvica ne vsebuje negativnih vrednosti zato pravimo, da je absolutna temperaturna lestvica.($0K =-273^{\circ}$ in $0^{\circ} =-273K$)\\
%\begin{center}
%	\includegraphics[width=15cm, height=15cm,keepaspectratio=true]{Temperatura.png}
%\end{center}
\begin{tikzpicture}
	\begin{axis}[
	    xlabel={T[$^{\circ} C$]},
	    ylabel={l[m]},
	    xmin=-293, xmax=546,
	    ymin=0, ymax=819,
	    xtick={-273,0,273,546},
	    xticklabels={0K,273K,546K, 819K},
	    ytick={273,546},
	    yticklabels={$l_1$,$l_2$},
	    ymajorgrids=true,
	    xmajorgrids=true,
	    grid style=dashed,
	    axis lines=middle,
	]
	 
	\addplot[domain=0:546,red] {x+273};
	\addplot[domain=-293:0,red, dashed] {x+273};
	\end{axis}
\end{tikzpicture}\\
V kolikšnem razmerju je temperatura s kinetično energijo $\rightarrow$ v linearnem.\\
%\foreach \n[count=\i] in {1,...,5}{ To je i: \i in to je rez \pgfmathparse{(\i-1) * 5} \pgfmathresult}\\
\begin{tikzpicture}
	\begin{axis}[
	    xlabel={t[s]},
	    ylabel={$W_k$},
	    xmin=0, xmax=10,
	    ymin=0, ymax=5,
	    xtick={0,2,4,6,8,10},
	    ytick={0,1,2,3,4,5},
	    yticklabels={0,1,2,$\overline{W_k}$,4},
	    ymajorgrids=true,
	    xmajorgrids=true,
	    grid style=dashed,
	    axis lines=middle,
	]
	\addplot[domain=0:10,red, dashed] {3};
	\addplot[domain=0:1,red] {2};
	\addplot[domain=1:3,red] {4};
	\addplot[domain=3:6,red] {2};
	\addplot[domain=6:10,red] {4};
	\end{axis}
\end{tikzpicture}\\
\begin{align*}
	{\color{bostonuniversityred}\overline{W_k}} &= {\color{bostonuniversityred}\frac{\overline{\mu}\overline{v}^2}{2}}\\
	\mu & \dots \text{masa molekule}\\
\end{align*}
Hitrost molekule se spreminja s korenom od časa.\\
Termometri izkoriščajo to, da se s temperaturo veča in manjša prostornina snovi:\\
\begin{itemize}
	\item kapljevinski(alkoholni, plinski)
	\item uporovni(nižja temperatura, večji upor)
	\item bimetalni(iz dveh različnih kovin, ki se različno raztezajo) $\rightarrow$ ko se dovolj raztegne prekine električni krog in izklopi napravo
\end{itemize}
%\begin{center}
%	\includegraphics[width=15cm, height=15cm,keepaspectratio=true]{Temperatura2.png}
%\end{center}