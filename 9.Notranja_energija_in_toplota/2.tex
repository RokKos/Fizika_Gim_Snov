{\color{indiagreen}\subsection{Specifična toplota}}
Načini segrevanja:
\begin{itemize}
	\item \textbf{Pri $V = konst.$}
	\begin{align*}
		\Delta W_n &= m c_v \Delta T\\
		c_v &\dots \text{specifična toplota pri konstatnem volumnu}\\
	\end{align*}
	\item \textbf{Pri $p = konst.$}
	\begin{align*}
		Q &= m c_p \Delta T\\
		c_p &\dots \text{specifična toplota pri konstatnem tlaku}\\
		A = -p \Delta V &\dots \text{volumen se veča in odriva okolico in s tem povzroča delo}\\
		\Delta W_n &= Q + A\\
		m c_v \Delta T &= m c_p \Delta T - p \Delta V /*\frac{1}{m \Delta T}\\
		c_v &= c_p \frac{p \Delta V}{m \Delta T}\\
		c_p &> c_v\\
	\end{align*}
\end{itemize}
\textbf{Ker če se segreva pri stalnem tlaku se snov segreva in opravi delo.}