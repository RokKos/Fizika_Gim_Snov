{\color{indiagreen}\subsection{Energijski zakon}}
$W_n = W_k$(termično gibanje)$+ W_p$(vezi med mulekulami)$ + W_p$(posameznega delca)\\
Idealni plin(model) sestavljajo točkaste molukule, idelano prožno trkajo, zanemarimo vezi med molekulami in notranje energije delcev.\\
\begin{align*}
	{\color{bostonuniversityred}W_n} &= {\color{bostonuniversityred}N\overline{W_k}}\\
	N &\dots \text{število delcev}\\
	N &= \frac{m}{\mu}\\
	\mu &\dots \text{masa molekule}\\
	\mu &= M * u\\
	u &= 1,66 * 10^{-27}kg\\
	\overline{W_k} &= \frac{3}{2}kT\\
	W_n &= \frac{m}{Mu}\frac{3}{2}kT\\
	W_n &= m\frac{3k}{2Mu}T\\
	c &\dots\text{specifična toplota}\\
	c &= \frac{3k}{2Mu}\\
	{\color{bostonuniversityred}W_n} =& {\color{bostonuniversityred}mcT}\dots\text{absolutna vrednost notranje energije}\\
	{\color{bostonuniversityred}\Delta W_n} =& {\color{bostonuniversityred}mc\Delta T}\dots\text{sprememba notranje energije}\\
	{\color{bostonuniversityred}c} = {\color{bostonuniversityred}\frac{\Delta W_n}{m\Delta T}[1\frac{J}{kgK}]}&\text{koliko energije potrbujemo, da 1 kg snovi sefrejemo za 1 Kelvin}\\
	Q &\dots \text{toplota}
\end{align*}
Toplota je del notranje energije, ki se ob toplotnem stiku pretaka iz telesa z višjo temperaturo v telo z nižjo temperaturo.\\
\begin{align*}
	{\color{bostonuniversityred}W_n} &= {\color{bostonuniversityred}A + Q}\dots \text{energijski zakon termodinamike}\\
\end{align*}
Če je $A = 0, \Delta W_n \rightarrow Q = mc\Delta T$\\
Če je $Q + 0, \Delta W_n = A$ (je toplotno izolirano)\\