{\color{indiagreen}\subsection{Toplotni tok}}
\begin{align*}
	P &= \frac{Q}{t}[\frac{J}{s}=1W]\\
\end{align*}
Tok toplote, ki se skozi dan presek pretoči v določenem času\\
\begin{align*}
	j &= \frac{P}{S}[1\frac{W}{m^2}]\\
	j &\dots \text{gostota toplotnega toka}
\end{align*}
Kolikšen toplotni tok se pretaka skozi izbran presek\\
%\begin{center}
%	\includegraphics[width=15cm, height=15cm,keepaspectratio=true]{ToplotniTok.png}
%\end{center}
\begin{align*}
	{\color{bostonuniversityred}P} &= {\color{bostonuniversityred}\frac{\gamma S \Delta T}{d}}\\
	\gamma &\dots \text{toplotna prevodnost}\\
	\gamma &= \frac{pd}{S \Delta T}[1 \frac{Wm}{m^2 K} = 1 \frac{W}{mK}]\\
\end{align*}
Toplotni tok, ki se s časom ne spreminja pravimo stacionarni toplotni tok.\\
\begin{align*}
	P &= \frac{\Delta T}{\frac{d}{\gamma S}}\\
	R &= \frac{d}{\gamma S}[1\frac{m^2K}{Wm^2} = 1\frac{K}{W}]\dots \text{toplotni upor}\\	
	P &= \frac{\Delta T}{R}\\
\end{align*}
Snovi:
\begin{itemize}
	\item toplotni izolatorji(stiropor, volna \dots) \textbf{R večji}
	\item toplotni prevodniki(baker, kovine \dots) \textbf{R manjši}
\end{itemize}
\textbf{Večplastna stena}\\
%\begin{center}
%	\includegraphics[width=15cm, height=15cm,keepaspectratio=true]{Toplotna prevodnost.png}
%\end{center}
\begin{align*}
	P &= \frac{\Delta T}{R}\\
	R &= R_1 + R_2 + R_3\\
\end{align*}
Skozi plati teče enak toplotni upor.\\
\textbf{Stena z oknom}\\
%\begin{center}
%	\includegraphics[width=15cm, height=15cm,keepaspectratio=true]{Toplotna prevodnost2.png}
%\end{center}
\begin{align*}
	P = P_1 + P_2\\
\end{align*}


