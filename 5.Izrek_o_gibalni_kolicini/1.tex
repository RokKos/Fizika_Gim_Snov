{\color{indiagreen}\subsection{Sunek sile in gibalna količina}}
\begin{align*}
	\vec{F} &= m\vec{a}\\
	\vec{a} &= \frac{\Delta\vec{v}}{\Delta t} = \frac{\vec{v_2}-\vec{v_1}}{\Delta t}\\
	\vec{F} &= m\frac{\vec{v_1}-\vec{v_2}}{\Delta t}\\
	\vec{F}\Delta t &= m\vec{v_1}-\vec{v_2} \rightarrow izrek o gibalni kolicini\\
	\vec{G} &= m\vec{v} \dots Gibalna kolicina [Ns, \frac{kgm}{s}]\\
	\vec{F}\Delta t &= \vec{G}_2 - \vec{G}_1 = \Delta\vec{G}\\
\end{align*}
\textbf{Izrek o ohranitvi energije}
Če je $\vec{F}\Delta t = 0 \rightarrow \Delta\vec{G} \rightarrow \vec{G}_2 = \vec{G}_1$.\\
Če je sunek vseh zunanjih sil enak nič potem se gibalna količina sistema ohrani.