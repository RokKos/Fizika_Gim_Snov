{\color{indiagreen}\subsection{Hookov zakon}}
l \dots prvotna dolžina\\
x \dots raztezek\\
S \dots premer žice\\
\begin{align*}
	\frac{F}{S} = \Delta\\
\end{align*}
$\Delta$ \dots raztezna napestost [$\frac{N}{m^2}$]\\
\begin{align*}
	\frac{x}{k} = \epsilon\\
\end{align*}
$\epsilon$ \dots relativni raztezek\\
\textbf{Hookov zakon}:
\begin{align*}
	{\color{bostonuniversityred}\frac{F}{S} = E\frac{x}{l}}\\
	F = \frac{ES}{l}x\\
	{\color{bostonuniversityred}F = kx}\\
	{\color{bostonuniversityred}k = \frac{ES}{l}}\\ 
\end{align*}
E \dots prožnostni model snovi [$\frac{N}{m^2}$]

