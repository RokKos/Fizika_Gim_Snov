{\color{indiagreen}\subsection{Navor}}
M \dots navor [1Nm]\\
\begin{align*}
	{\color{bostonuniversityred}M = rF''}\\
	F'' = F\cos\alpha\\
	M = rF\cos\alpha\\
	\cos\alpha = \frac{r'}{r}\\
	M = rF\frac{r'}{r}\\
	{\color{bostonuniversityred}M = Fr'}\\
\end{align*}
r' \dots ročica(pravokotna razdalja med nosilko sile in osjo)\\
\begin{align*}
	\vec{M} = \vec{r} X \vec{F}\\
\end{align*}
\textbf{Navor} je ročica krat sila. \textbf{Smer navora} je po \underline{desnem vijaku}(v našem primeru bi kazal v list). Mi bomo gledali samo kako navor zasuka telo.\\
\textbf{Izrek o ravnovesju} pravi:
\begin{enumerate}
	\item Da mora biti \textbf{rezultanta} vseh \textbf{zunanjih sil 0}
	\item Da mora biti \textbf{rezultanta} vseh \textbf{navorov 0}
\end{enumerate}
Takrat telo miruje ali se giba premo enakomerno.