{\color{indiagreen}\subsection{Transformator}}
Električno energijo pretvori nazaj v električno energijo, samo pri različni napetosti in različnem toku.\\
%\begin{center}
%	\includegraphics[width=15cm, height=15cm,keepaspectratio=true]{Transformator.png}
%\end{center}
\begin{align*}
	\text{\RomanNumber{1}} &\dots \text{Primarna tuljava}\\
	\text{\RomanNumber{2}} &\dots \text{Sekundarna tuljava}\\
	n_1 &\dots \text{Število ovojev primarne tuljave}\\
	n_2 &\dots \text{Število ovojev sekundarne tuljave}\\
\end{align*}
Deluje na osnovi indukcije. Z njim lahko transformiramo samo izmenične napetosti.\\
\begin{align*}
	U_1 &= n_1 \frac{\Delta \Phi}{\Delta t}\\
	U_2 &= n_2 \frac{\Delta \Phi}{\Delta t}\\
	{\color{bostonuniversityred}\frac{U_1}{U_2}} &= {\color{bostonuniversityred}\frac{n_1}{n_2}} \dots \text{Razmerje napetosti je enako razmerju med ovoji}\\
	P_1 &= P_2 \dots \text{Če ni izgub, transformator deluje z zelo velikim izkoristkom(90\%)}\\
	U_1 I_1 &= U_2 I_2\\
	\frac{I_1}{I_2} &= \frac{U_1}{U_2} = \frac{n_1}{n_2}\\
	{\color{bostonuniversityred}\frac{I_1}{I_2}} &= {\color{bostonuniversityred}\frac{n_1}{n_2}}\\
\end{align*}
Bolj daleč kot je transformator večja je napetost in manjši je tok, da pride do mnjših izgub.\\
\begin{align*}
	P &= R I_{ef}^2 \dots \text{Čim manjša efektivna napetost}\\
	Q &= Pt\\
\end{align*}
{\color{bostonuniversityred}Vrtinčni tokovi} povzročajo izgube, ker transformator segrevajo.\\