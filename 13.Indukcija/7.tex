{\color{indiagreen}\subsection{Električni nihajni krog}}
%\begin{center}
%	\includegraphics[width=15cm, height=15cm,keepaspectratio=true]{ElNihajniKrog.png}
%\end{center}
Zaprt električni krog(sestavljata ga tuljava in kondenzator). Kondenzator je nabit.\\
$t = 0$\\
%\begin{center}
%	\includegraphics[width=15cm, height=15cm,keepaspectratio=true]{ElNihajniKrog2.png}
%\end{center}
$e_0, E_0 \rightarrow$ največji naboj in električno polje\\
$I = 0, B = 0 \rightarrow$ v tuljavi ni toka in ni magnetnega polja\\
Žačne teči tok\\
$t = \frac{t_0}{4}$\\
%\begin{center}
%	\includegraphics[width=15cm, height=15cm,keepaspectratio=true]{ElNihajniKrog3.png}
%\end{center}
$e = 0, E = 0 \rightarrow$ na kondenzatorju\\
$I = 0, B = 0 \rightarrow$ največji tok in magnetno polje\\
Tok se začne manjšati in tuljava se začne upira(inducirana napetost)\\
$t = \frac{t_0}{2}$\\
%\begin{center}
%	\includegraphics[width=15cm, height=15cm,keepaspectratio=true]{ElNihajniKrog4.png}
%\end{center}
$e_0, E_0$\\
$I = 0, B = 0$\\
$t = \frac{3t_0}{4}$\\
%\begin{center}
%	\includegraphics[width=15cm, height=15cm,keepaspectratio=true]{ElNihajniKrog5.png}
%\end{center}
$e = 0, E = 0$\\
$I_0, B_0$\\
$t = t_0$\\
%\begin{center}
%	\includegraphics[width=15cm, height=15cm,keepaspectratio=true]{ElNihajniKrog6.png}
%\end{center}
$e_0, E_0$\\
$I = 0, B = 0$\\
Ponavljanje dejanja(periodničnost) spreminjajo se vrednosti.\\