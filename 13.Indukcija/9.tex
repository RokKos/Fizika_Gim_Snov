{\color{indiagreen}\subsection{Elektro magnetno valovanje}}
%\begin{center}
%	\includegraphics[width=15cm, height=15cm,keepaspectratio=true]{ElektroMagnetnoValovanje.png}
%\end{center}
Dipolna antena(odprt električni nihajni krog)\\
%\begin{center}
%	\includegraphics[width=15cm, height=15cm,keepaspectratio=true]{ElektroMagnetnoValovanje2.png}
%\end{center}
%\begin{center}
%	\includegraphics[width=15cm, height=15cm,keepaspectratio=true]{ElektroMagnetnoValovanje.3png}
%\end{center}
\begin{align*}
	\vec{B} \perp \vec{E} &, \vec{B} \perp \vec{c}, \vec{c} \perp \vec{E}\\
\end{align*}
To je {\color{bostonuniversityred}transverzalno valovanje}.\\
\begin{align*}
	\omega_m &= \frac{B_0^2}{2\mu_0}\\
	\omega_e &= \frac{\varepsilon_0 E_0^2}{2}\\
	\frac{B_0^2}{\cancelto{}{2}\mu_0} &= \frac{\varepsilon_0 E_0^2}{\cancelto{}{2}}\\
	E_0^2 &= \frac{1}{\mu_0 \varepsilon_0} B_0^2\\
	E_0 &= \frac{1}{\sqrt{\mu_0 \varepsilon_0}} B_0\\
	{\color{bostonuniversityred}c} &= {\color{bostonuniversityred}\frac{1}{\sqrt{\mu_0 \varepsilon_0}}} \dots \text{hitrost svetlobe ali elektromagnetnega valovanja}\\
	{\color{bostonuniversityred}E_0} &= {\color{bostonuniversityred}c B_0}\\
\end{align*}
