{\color{indiagreen}\subsection{Indukcija pri premikanju vodnika v magnetnem polju}}
%\begin{center}
%	\includegraphics[width=15cm, height=15cm,keepaspectratio=true]{Indukcija.png}
%\end{center}
%\begin{center}
%	\includegraphics[width=15cm, height=15cm,keepaspectratio=true]{Indukcija2.png}
%\end{center}
\begin{align*}
	E_i &\dots \text{Inducirano električno polje}\\
	\vec{F_e} = \vec{F_m} &\dots \text{zato, ker se elektroni po žici ne premikajo pospešeno}\\
	eE_i &= evB\\
	E_i &= vB /*b\\
	bE_i &= vBb\\
	{\color{bostonuniversityred}U_i} &= {\color{bostonuniversityred}vBb} \dots \text{inducirana napetost, $\vec{v} \perp \vec{B}$}\\
\end{align*}
Če premikamo vodnik po magnetnem polju se na njegovih robovih pojavi inducirana napetost.\\
\begin{align*}
	{\color{bostonuniversityred}\vec{F_{l}}} &= {\color{bostonuniversityred}I \vec{b} \times \vec{B}}\\
\end{align*}
\textbf{LENZOVO PRAVILO}\\
Inducirana napetost požene induciran tok vedno v takšno smer, da nastala magnetna sila na vodnik nasprotuje premikanju vodnika.\\
\\
Mehansko delo spreminja v električno, ker magnetna sila vedno nasprotuje smeri premikanja\\
Če je hitrost vzporedna silnicam magnetnega polja potem je $U_i = 0(\vec{v} \perp \vec{B})$\\
\begin{align*}
	U_i &= vbB\\
	\Delta S &= vb\Delta t\\
	vb &= \frac{\Delta S}{\Delta t}\\
	U_i &= \frac{\Delta SB}{\Delta t}\\
	U_i &= -\frac{\Phi}{\Delta t}\dots \text{Faradejev zakon indukcije}\\
\end{align*}
Inducirana napetost je spremeba magnetnega pretoka v danem času. Zaradi Lenzovega pravila je predznak minus\\
\begin{align*}
	U_i \Delta t &= \Delta \Phi \dots \text{sunek napetosti je enak spremebi magnetnega polja}\\
\end{align*}
%\begin{center}
%	\includegraphics[width=15cm, height=15cm,keepaspectratio=true]{Indukcija3.png}
%\end{center}
\begin{align*}
	\Delta \Phi &= BS\\
\end{align*}
%\begin{center}
%	\includegraphics[width=15cm, height=15cm,keepaspectratio=true]{Indukcija4.png}
%\end{center}
\begin{align*}
	\Delta \Phi &= 2BS\\
\end{align*}
\textbf{Tuljava}\\
%\begin{center}
%	\includegraphics[width=15cm, height=15cm,keepaspectratio=true]{Indukcija5.png}
%\end{center}
\begin{align*}
	\Phi &= NBS\\
\end{align*}
Smer toka obrnemo:\\
\begin{align*}
	\Phi &= -NBS\\
	{\color{bostonuniversityred}\Delta \Phi} &= {\color{bostonuniversityred}2NBS}\\
\end{align*}

