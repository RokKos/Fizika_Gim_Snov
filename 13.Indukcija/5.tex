{\color{indiagreen}\subsection{Induktivnost tuljave}}
%\begin{center}
%	\includegraphics[width=15cm, height=15cm,keepaspectratio=true]{InduktivnostTuljave.png}
%\end{center}
\begin{align*}
	B &= \frac{\mu_0 nI}{l}\\
	\Phi &= nBs\\
	\Phi &= \frac{\mu_0 n^2 IS}{l}\\
	U_i = \frac{\Delta \Phi}{\Delta t} &= \frac{\mu_0 n^2 IS}{l} \frac{\Delta I}{\Delta t} \dots \text{Inducirana napetost}\\
	L &= \frac{\mu_0 n^2 S}{l} \dots \text{Induktivnost tuljave}\\
	{\color{bostonuniversityred}U_i} &= {\color{bostonuniversityred}L \frac{\Delta I}{\Delta t}} [1\frac{Vs}{A} = 1H] \dots \text{{\color{bostonuniversityred}Henry}}\\
\end{align*}
Inducirana napetost pove kolikšna napetost se inducira, če se tok spremeni za 1A v 1 sekundi.\\
