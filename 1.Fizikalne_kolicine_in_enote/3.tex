{\color{indiagreen}\subsection{Merjenje}}
\textbf{NAPAKE}:
\begin{itemize}
	\item SLUČAJNE(odvisne od natačnosti merilca) $\rightarrow$ te napake se da zmanjašati z \underline{večkratnim merjenjem}
	\item SISTEMATIČNE(odvisne od merilne naprave) $\rightarrow$ se jih \underline{neda odpraviti} z večkratnim merjenjem
\end{itemize}

Vse meritve zapišemo v \textbf{tabelo}

\begin{center}
	\begin{tabular}{|c c|}
		\hline 
		dolžina l & [m]\\
	 	\hline
	 	1 & $x_1$\\
	 	2 & $x_2$\\
	 	3 & $x_3$\\
	 	\vdots & \vdots\\
	 	n & $x_n$\\
	 	\hline
 	\end{tabular}
\end{center}

Izračun povprečne vrednosti : $\overline{x}$

\begin{align*}
	{\color{bostonuniversityred}{\overline{x} = \frac{x_1 + x_2 + \ldots + x_n}{n}}}
\end{align*}

\textbf{Absolutna Napaka $\Delta x$}\\

$\Delta x$ je največje odstopanje meritve od povprečne vrednosti.

\begin{align*}
	{\color{bostonuniversityred}{x = \overline{x} \pm \Delta x}}
\end{align*}

\textbf{Relativna Napaka $\delta x$}\\

\begin{align*}
	{\color{bostonuniversityred}{\delta x =\frac{\Delta x}{\overline{x}}}}\\
	\\
	{\color{bostonuniversityred}{x = \overline{x}(1 \pm \frac{\Delta x}{\overline{x}})}}
\end{align*}




