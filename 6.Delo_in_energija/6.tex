{\color{indiagreen}\subsection{Ohranitev kinetične in potencialne energije}}
\begin{align*}
	A &= A_t + A_o\\
	A &= \frac{mv_2^2}{2} - \frac{mv_1^2}{2} \dots \text{delo vseh zunanjih sil}\\
	A_t &= mgz_1 - mgz_2 \dots \text{delo vseh zunanjih sil}\\
	A_o & \text{$\dots$ delo vseh zunanjih sil razen teže}\\
	A_o &= A - A_t\\
	{\color{bostonuniversityred}A_o} &= {\color{bostonuniversityred}\Delta W_k\Delta W_p}\\
\end{align*}
Zraven ni delo teže, ker smo ga upoštevali pri potencialni energiji.\\
Če je $A_o = 0$, na telo deluje le teža.\\
\begin{align*}
	0 &= \Delta W_k \Delta W_p\\
	{\color{bostonuniversityred}\Delta W_k \Delta W_p} &= konst. \text{Izrek o ohranitvi $W_k$ in $W_p$}\\
\end{align*}
Če na telo deluje samo teža se ohranja vsota potencialne in kinetične energije.\\